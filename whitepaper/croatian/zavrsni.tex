\documentclass[times, utf8, zavrsni]{fer}
\usepackage{booktabs}
\usepackage{listings}
\graphicspath{ {images/} }

\begin{document}

\thesisnumber{000}

\title{Sustav za izdvajanje tabličnih podataka iz web-stranica<Paste>}

\author{Stanko Krtalić Rusendić}

\maketitle

\zahvala{}

\tableofcontents

\chapter{Uvod}

\bibliography{zavrsni}
\bibliographystyle{fer}

\begin{sazetak}
Proučiti i opisati dostupne sustave za izdvajanje tabličnih podataka iz
polu-strukturiranih dokumenata kao što su web-stranice ili PDF dokumenti.
Osmisliti i ostvariti sustav za izdvajanje strukturiranih tabličnih podaka iz
navedenih dokumenata s naglaskom na web-stranice.
Sustav treba izložiti svoje funkcionalnosti koristeći sučelje naredbenog retka
za koje je potrebno osmisliti skup naredbi za jednostavni interaktivni rad.
Nadalje, ostvariti i osnovno programsko sučelje sustava. Prikupiti skup
web-stranica za ispitivanje rada sustava te ocjeniti uspješnost izdvajanja
tabličnih podataka s obzirom na složenost strukture ulaznog dokumenta. Opisati
izgrađeni sustav, navesti upute za postavljanje, načine korištenja, navesti
literaturu i primljenu pomoć.

\kljucnerijeci{}
\end{sazetak}

\engtitle{System for Tabular Data Extration from Web-pages}
\begin{abstract}
Study and describe available systems for extraction of tabular data from
semi-structured documents like web sites or PDF documents. Think of and
implement a system for extracting of structured tabular data from the types of
documents mentioned above with an emphesys on web sites.
The system should be implemented as a command line interface with a set of
commands for interactive use. Aggregate a set of web sites for testing the
effectivnes of data extraction regarding the com[plexety of the input.
Describe the created system, write setup instrunctions, usage instructions,
cite the literature used and any help received.

\keywords{}
\end{abstract}

\end{document}
