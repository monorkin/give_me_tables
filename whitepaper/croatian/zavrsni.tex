\documentclass[times, utf8, zavrsni]{fer}
\usepackage{booktabs}
\usepackage{listings}
\graphicspath{ {images/} }

\begin{document}

\thesisnumber{000}

\title{Sustav za izdvajanje tabličnih podataka iz web-stranica}

\author{Stanko Krtalić Rusendić}

\maketitle

\zahvala{}

\tableofcontents

\chapter{Uvod}



\chapter{Postojeća rješenja}

Postoji nekolicina javno dostupnih gotovih rješenja za izdvajanje tabličnih
podataka iz PDF i HTML dokumenata.

\section{Tabula \cite{tabula_repository}}

Tabula je rješenje otvorenog koda napisano u programskom jeziku JRuby.
Nudi grafičko sučelje tako sto pokrene server preko kojeg servira web
aplikaciju i sučelje za komandnu liniju kroz JAR datoteku. Moguće je vaditi
tablične podatke jedino iz PDF dokumenata koji su tekstualnog tipa, odnosno,
ne posjeduje mogućnosti prepoznavanja teksta i tablica iz slika.

JRuby je verzija programskog jezika Ruby namijenjena izvođenju na Java
virtualnom stroju (eng. kratica JVM). To JRubyju omogućuje korištinje
programskih biblioteka iz ostalih jezika koji se izvršavaju na JVM-u. Tako
Tabula za parsiranje PDF dokumenata koristi biblioteke JPedal i Apache
PDFToolbox napisane u programskom jeziku Java.

Interno, aplikacija pretražuje PDF dokument za tabličnim objektima koje potom
pretvara u CSV zapis i pohranjuje kao datoteku.

Najveće manjkavosti ovog rješenja su ne-interaktivno sučelje komandne linije,
činjenica da se izvodi kao zaseban proces (server) sto zahtijeva od korisnika
da koristi internet preglednik i ovisnost o lokalnoj instalaciji JVM-a.

\section{pdftabextract \cite{pdftabextract_repository}}

pdftabextract je rješenje otvorenog koda napisano u programskom jeziku Python.
Nudi isključivo sučelje komandne linije i ne može direktno obrađivati PDF
datoteke, već ih je prethodno potrebno konvertirati u pdf2xml format.

Korištenje programskog alata pdf2xml omogućava ovom rješenju reparaciju
podataka prije nego li ih krenemo obrađivati. Moguće reparacije su rotacija
stranice, ispravka iskrivljene slike i razdvajanje dvostrukih stranica. Ovaj
korak je iznimno bitan kako rješenje ne konzumira strukturirane podatke već
pokušava otkriti grupe podataka koje su vizualno organizirane u redove i
stupce, te ih organizirati u CSV datoteku.

Takav pristup obradi PDF datoteka omogućava pdftabextractu rad s dokumentima
koji su bili konvertirani iz slike u text, te dokumente koji nemaju strogo
definiran tablični format. Međutim nedostatak ovakvog pristupa je nepouzdanost.
CSV datoteke, koje ovo rješenje vraća kao rezultat, često izgledaju ne
strukturirano i potrebno je ručno sanitizirati rezultat. Također, iz testiranja
sam ustanovio da često krivo zaključi koji podaci pripadaju retku tablice.
Tako će, na primjer, podatak koji zauzima vise redaka unutar jedne ćelije
tablice biti pretvoren u vise ćelija u vise redaka.

\section{PDFFigures 2.0 \cite{pdffigures_2_repository}}

\section{PDFLayoutTextStripper \cite{pdflayouttextstripper_repository}}



\chapter{Idejno rješenje}

\section{Uvod}

\section{Željena funkcionalnost}

\section{Željeno ponašanje}

\section{Arhitektura aplikacije}



\chapter{Ostvareno rješenje}

\section{Korištene tehnologije}

\section{Korištene programske biblioteke}

\section{Uputstva za korištenje rješenja}



\chapter{Zaključak}

\section{Moguća unapređenja}

\section{Publikacija rješenja}



\bibliography{zavrsni}
\bibliographystyle{fer}

\begin{sazetak}

Proučiti i opisati dostupne sustave za izdvajanje tabličnih podataka iz
polu-strukturiranih dokumenata kao što su web-stranice ili PDF dokumenti.
Osmisliti i ostvariti sustav za izdvajanje strukturiranih tabličnih podaka iz
navedenih dokumenata s naglaskom na web-stranice.
Sustav treba izložiti svoje funkcionalnosti koristeći sučelje naredbenog retka
za koje je potrebno osmisliti skup naredbi za jednostavni interaktivni rad.
Nadalje, ostvariti i osnovno programsko sučelje sustava. Prikupiti skup
web-stranica za ispitivanje rada sustava te ocjeniti uspješnost izdvajanja
tabličnih podataka s obzirom na složenost strukture ulaznog dokumenta. Opisati
izgrađeni sustav, navesti upute za postavljanje, načine korištenja, navesti
literaturu i primljenu pomoć.

\kljucnerijeci{}
\end{sazetak}

\engtitle{System for Tabular Data Extration from Web-pages}
\begin{abstract}

Study and describe available systems for extraction of tabular data from
semi-structured documents like web sites or PDF documents. Think of and
implement a system for extracting of structured tabular data from the types of
documents mentioned above with an emphesys on web sites.
The system should be implemented as a command line interface with a set of
commands for interactive use. Aggregate a set of web sites for testing the
effectivnes of data extraction regarding the com[plexety of the input.
Describe the created system, write setup instrunctions, usage instructions,
cite the literature used and any help received.

\keywords{}
\end{abstract}

\end{document}
