\documentclass[times, utf8, zavrsni]{fer}
\usepackage{booktabs}
\usepackage{listings}
\graphicspath{ {images/} }

\begin{document}

\thesisnumber{000}

\title{Sustav za izdvajanje tabličnih podataka iz web-stranica}

\author{Stanko Krtalić Rusendić}

\maketitle

\zahvala{}

\tableofcontents

\chapter{Uvod}



\chapter{Postojeća rješenja}

\section{Uvod}

Postoji nekolicina, javno dostupnih, gotovih rješenja za izdvajanje tabličnih
podataka iz PDF \cite{pdf_documentation}
i HTML \cite{html_documentation}
datoteka. Svako od rješenja ima prednosti i mane koje je
bitno proučiti prije razrade i implementacije vlastitog rješenja.

\section{Tabula}

Tabula \cite{tabula_repository} je rješenje otvorenog koda napisano u
programskom jeziku JRuby \cite{jruby_documentation}.
Nudi grafičko sučelje tako sto pokrene server preko kojeg servira web
aplikaciju i sučelje za komandnu liniju kroz
JAR \cite{java_language_specification} datoteku. Moguće je vaditi
tablične podatke jedino iz PDF dokumenata koji su tekstualnog tipa, odnosno,
ne posjeduje mogućnosti prepoznavanja teksta i tablica iz slika.

JRuby je verzija programskog jezika Ruby namijenjena izvođenju na Java
virtualnom stroju \cite{java_virtual_machine_specification} (eng. kratica JVM).
To JRubyju omogućuje korištinje
programskih biblioteka iz ostalih jezika koji se izvršavaju na JVM-u. Tako
Tabula za parsiranje PDF dokumenata koristi biblioteke JPedal i Apache
PDFToolbox \cite{apache_pdftoolbox_specification} napisane u programskom
jeziku Java \cite{java_language_specification}.

Interno, aplikacija pretražuje PDF dokument za tabličnim objektima koje potom
pretvara u CSV zapis i pohranjuje kao datoteku.

Najveće manjkavosti ovog rješenja su ne-interaktivno sučelje komandne linije,
činjenica da se izvodigg kao zaseban proces (server) sto zahtijeva od korisnika
da koristi internet preglednik i ovisnost o lokalnoj instalaciji JVM-a.

\section{pdftabextract}

pdftabextract \cite{pdftabextract_repository} je rješenje otvorenog koda napisano
u programskom jeziku Python.
Nudi isključivo sučelje komandne linije i ne može direktno obrađivati PDF
datoteke, već ih je prethodno potrebno konvertirati u pdf2xml format.

Korištenje programskog alata pdf2xml omogućava ovom rješenju reparaciju
podataka prije nego li ih krenemo obrađivati. Moguće reparacije su rotacija
stranice, ispravka iskrivljene slike i razdvajanje dvostrukih stranica. Ovaj
korak je iznimno bitan kako rješenje ne konzumira strukturirane podatke već
pokušava otkriti grupe podataka koje su vizualno organizirane u redove i
stupce, te ih organizirati u CSV datoteku.

Takav pristup obradi PDF datoteka omogućava pdftabextractu rad s dokumentima
koji su bili konvertirani iz slike u text, te dokumente koji nemaju strogo
definiran tablični format.

Međutim nedostatak ovakvog pristupa je nepouzdanost.
CSV datoteke, koje ovo rješenje vraća kao rezultat, često izgledaju ne
strukturirano i potrebno je ručno sanitizirati rezultat. Također, iz testiranja
sam ustanovio da često krivo zaključi koji podaci pripadaju retku tablice.
Tako će, na primjer, podatak koji zauzima vise redaka unutar jedne ćelije
tablice biti pretvoren u vise ćelija, tj. redaka u tablici.

\section{PDFFigures 2.0}

PDFFigures 2.0 \cite{pdffigures_2_repository} je rješenje otvorenog koda
napisano u programskom jeziku Scala \cite{scala_documentation}. Slično kao
pdftabextract nudi isključivo
sučelje komandne linije, ali može direktno procesirati PDF datoteke. Fokus
rješenja je na ekstrahiranju podataka iz znanstvenih radova u polju računalne
znanosti.

Rezultat koji ovo rješenje vraća je set podataka koji opisuju figure PDF
dokumenta. Sve sto nije isključivo kvadrat s tekstom se interpretira kao
figura, npr. slike, tablice, ilustracije i slično.

Manjkavosti ovog rješenja slične su onima od Tabule \cite{tabula_repository}.
Za pokretanje je potrebno
imati instaliraju točnu verziju JVM-a, ne postojanje interaktivnog sučelja i
potreba za dodatnim eksternim programskim bibliotekama. Također, sučelje
komandne linije je nezgrapno za korištenje kako se svodi na pisanje Scala koda
koji se predaje Java virtualnom stroju na izvođenje.

\section{PDFLayoutTextStripper}

PDFLayoutTextStripper \cite{pdflayouttextstripper_repository} je programska
biblioteka otvorenog koda napisano u programskom jeziku Java.

Za razliku od prethodno navedenih rješenja primarna svrha ovog rješenja nije
ekstrahiranje tabličnih podataka, već ekstrahiranje teksta i očuvanje njegovog
prostornog rasporeda.

Ima sposobnost direktne obrade PDF datoteka (sto uspijeva pomoću biblioteke iz
Apache PDFToolboxa).

Ova biblioteka ima iste nedostatke kao i Tabula \cite{tabula_repository} i
PDFFigures 2.0 \cite{pdffigures_2_repository}, te dodatno ne
nudi nikakvo sučelje za direktno korištenje već je potrebno implementirati
vlastito programsko sučelje.



\chapter{Idejno rješenje}

\section{Uvod}

Iz pregleda postojećih rješenja moguće je izdvojiti tri zajedničke mane.

\begin{itemize}
  \item Sva rješenja zahtijevaju da korisnik ima instaliran neki virtualni
    stroj i dodatne  programske biblioteke
  \item Nedostatak kvalitetnog i univerzalnog sučelja (bilo grafičkog ili iz
    komandne linije)
  \item Ne praćenje UNIX filozofije \cite{unix_philosophy}
\end{itemize}

pdftabextract donekle prati UNIX filozofiju i koristiti drugu aplikaciju kako bi
pretvorio PDF datoteke u pdf2xml format koji se lako parsira i obrađuje.

HTML datoteke je lakše parsirati i obrađivati od PDF datoteka kako je njihov
sadržaj strukturiran i jasno definiran.
Stoga bi konačno rješenje, u svrhu
jednostavnosti izvedbe i poboljšanja rezultata, trebalo isključivo
implementirati podršku za obradu HTML datoteka.

Za obradu PDF datoteka, ili bilo kojeg drugog tipa datoteke,
potrebno je istu prethodno pretvoriti u HTML datoteku pomoću alata poput
pdf2htmlEX \cite{pdf2htmlex_repository}.

\section{Željena funkcionalnost}

\subsection{Podržani tipovi ulaznih podataka}

Prethodno su navedene prednosti korištenja podataka zapisanih u HTML formatu.
Konačno rješenje ne bi smjelo biti limitirano isključivo na HTML podatke već
bi trebalo nuditi mogućnost za lakom nadogradnjom novih formata za slučaj
pojavljivanja formata koji se ne može pretvoriti u HTML ili za slučaj da se
uvidi da je efikasnije direktno podržavati neki format.

Rješenje mora moći primiti podatke direktno preko standardnog ulaza, pročitati
ih iz datoteke ili ih preuzeti s poveznice. Čitanje podataka iz datoteke i
preuzimanje s poveznice ne prati UNIX filozofiju, ali povećava ergonomičnost
rješenja. UNIX filozofija naglašava da bi za obradu podatak iz datoteka i s
poveznica trebali koristiti standardne alate poput cata, seda, curla i wgeta
kako bi predali sadržaj datoteke ili poveznice našem rješenju na standardni
ulaz. Međutim iz iskustva se pokazalo da takvo ulančavanje može stvoriti
kompleksne lance naredbi za postizanje radnje koja je u bilo kojem programskom
jeziku izrazito jednostavno za postignut. Stoga, u svrhu ergonomije i boljeg
korisničkog iskustva, će konačno rješenje podržavati i ne standardne ulaze
podataka.

\subsection{Sučelje komandne linije}

Kroz sučelje komandne linije korisnik može izdvojiti pojedinačnu tablicu i nad
njom izvršiti agregacijske funkcije, ili može izdvojiti sve tablice iz nekog
dokumenta. Time sto je izvoženje agregacijskih funkcija ograničeno na
pojedinačne tablice se smanjuje kompleksnost konačnog rješenja i olakšava se
rad korisniku kako ne mora učiti meta-jezik s kojim bi definirao izvođenje
željenih agregacijskih funkcija nad tablicama - problem uviđen u već postojećim
rješenjima, ali i u drugim alatima poput awk-a.

Osim navedenog mora se nuditi sučelje za izlistavanje tablica u dokumentu
kako bi korisnik mogao odrediti koje tablice želi izdvojiti. Sučelje bi
trebalo prikazati nekolicinu stupaca tablice (koliko korisniku stane na ekran)
i prva 3 retka.  Prethodno je spomenuto da mora postojati i sučelje za
1selekciju tablica koje se treba izdvojiti.

\subsection{Grafičko sučelje}

Kroz grafičko sučelje korisnik mora imati sposobnost napraviti sve sto je
moguće i u sučelju komandne linije. Grafičko sučelje bi trebalo predstavljati
interaktivni način rada za sučelje komandne linije. Ovakav pristup će konačno
rješenje učiniti nepristupačnim korisnicima koji ne znaju koristiti sučelje
komandne linije, međutim korisnicima koji znaju koristiti to sučelje će
olakšati proces učenja korištenja alata.

Postoji mogućnost da se stvori puno grafičko sučelje nad opisanim rješenjem.
Implementacija punog grafičkog sučelja bi trebala, preko sistemskog sučelja,
slati naredbe rješenju i koristiti povrate informacije za prikaz vizualnih
elemenata na ekranu. Alternativno, moguće je i razdvojiti jezgru ovog rješenja
od koda zaduženog za interakciju s korisnikom i tako rješenje pretvoriti u
programsku biblioteku. To bi omogućilo da se isti kod koristi da pogoni razna
sučelja prema korisniku.

\subsection{Podržani tipovi izlaznih podataka}

Rješenje mora nuditi mogućnost zapisa izlaznih podataka u raznim formatima
koji su standardni za tablične podatke - poput csv \cite{csv_documentation},
xls \cite{xls_documentation}, xlsx \cite{xlsx_documentation} i
ods \cite{ods_documentation} formata.
Slično argumentu s ulaznim datotekama, rješenje za izlazne datoteke mora nuditi
pohranu u datoteku i ispis na standardni izlaz. Ponovno, ovakva odluka ne
prati UNIX filozofiju, ali povećava ergonomičnost rješenja.

\section{Arhitektura aplikacije}


Kako bi konačno rješenje bilo nezavisno o nekom virtualnom stroju ili o
programskim bibliotekama potrebno ga je implementirati u programskom jeziku
koji se može kompajlirati u procesorski kod ili zapakirati u binarnu datoteku.

Programske biblioteke se moraju statično povezati u rezultantnu binarnu
datoteku aplikacije kako korisnik ne bi morao prilagođavati svoju programsku
okolinu i instalirati dodatne programske biblioteke da pokrene aplikaciju.

Dinamično povezivanje programskih biblioteka može kod korisnika uzrokovati
nepredviđene probleme poput ne kompatibilnosti dvaju ili vise biblioteka i
konflikt u verziji biblioteke između dvaju ili vise aplikacija. Osim toga,
krajnjeg korisnika se tereti da održava instalaciju programske biblioteke.

Rješenje se treba moći lako nadograđivati i proširivati. Prethodno je navedeno
postojanje mogućnosti da će rješenje morat podržavati ulazne datoteke
drugačijeg tipa od predviđenih HTML datoteka. Isto razmišljanje može se
primijeniti na izlazne datoteke, te na agregacijske funkcije. Kroz vrijeme će
se pokazati potreba za dodatnim agregacijskim funkcijama ili tipovima izlaznih
datoteka.

S toga konačno rješenje mora implementirati arhitekturu koja
podržava programske dodatke (eng. plugin).




\chapter{Ostvareno rješenje}

\section{Korištene tehnologije}

\section{Korištene programske biblioteke}

\section{Uputstva za instalaciju rješenja}

\section{Uputstva za korištenje rješenja}



\chapter{Zaključak}

\section{Moguća unapređenja}

\section{Publikacija rješenja}



\bibliography{zavrsni}
\bibliographystyle{fer}

\begin{sazetak}

Proučiti i opisati dostupne sustave za izdvajanje tabličnih podataka iz
polu-strukturiranih dokumenata kao što su web-stranice ili PDF dokumenti.
Osmisliti i ostvariti sustav za izdvajanje strukturiranih tabličnih podaka iz
navedenih dokumenata s naglaskom na web-stranice.
Sustav treba izložiti svoje funkcionalnosti koristeći sučelje naredbenog retka
za koje je potrebno osmisliti skup naredbi za jednostavni interaktivni rad.
Nadalje, ostvariti i osnovno programsko sučelje sustava. Prikupiti skup
web-stranica za ispitivanje rada sustava te ocjeniti uspješnost izdvajanja
tabličnih podataka s obzirom na složenost strukture ulaznog dokumenta. Opisati
izgrađeni sustav, navesti upute za postavljanje, načine korištenja, navesti
literaturu i primljenu pomoć.

\kljucnerijeci{}
\end{sazetak}

\engtitle{System for Tabular Data Extration from Web-pages}
\begin{abstract}

Study and describe available systems for extraction of tabular data from
semi-structured documents like web sites or PDF documents. Think of and
implement a system for extracting of structured tabular data from the types of
documents mentioned above with an emphesys on web sites.
The system should be implemented as a command line interface with a set of
commands for interactive use. Aggregate a set of web sites for testing the
effectivnes of data extraction regarding the com[plexety of the input.
Describe the created system, write setup instrunctions, usage instructions,
cite the literature used and any help received.

\keywords{}
\end{abstract}

\end{document}
